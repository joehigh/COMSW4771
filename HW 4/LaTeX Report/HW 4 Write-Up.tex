\documentclass[twoside,11pt]{homework}

\coursename{COMS 4771 Machine Learning (Spring 2020)} 

\studname{Joseph High}    % YOUR NAME GOES HERE
\studmail{jph2185@columbia.edu}% YOUR UNI GOES HERE
\hwNo{4}                   % THE HOMEWORK NUMBER GOES HERE
\date{\today} % DATE GOES HERE

\usepackage{graphicx}
%\usepackage{fancyhdr}
\usepackage{enumerate}
\usepackage{amsmath}
\usepackage{relsize}
\usepackage{mathtools}
\usepackage{dsfont}
\usepackage[dvipsnames]{xcolor}
\usepackage[makeroom]{cancel}
\usepackage{collectbox}
\usepackage{placeins}
%\usepackage{cleveref}
\usepackage{eqnarray}
%\usepackage{titlesec} 
\usepackage{bm} 
\usepackage{bbm}
\usepackage{hyperref}
\usepackage{flafter}
\usepackage{graphicx}
\usepackage{float}
%\titleformat{\subsubsection}[runin]
%  {\normalfont\normalsize\bfseries}{\thesubsubsection}{1em}{}

\DeclarePairedDelimiter{\2norm}{\lVert}{\rVert^2_2}
\newcommand{\defeq}{\vcentcolon=}
\newcommand\unif{\ensuremath{\operatorname{unif}}}
\DeclareMathOperator{\EX}{\mathbb{E}}% expected value
\newcommand{\1}[1]{\mathds{1}\left[#1\right]}

\DeclarePairedDelimiter{\infnorm}{\lVert}{\rVert_{\infty}}
\DeclarePairedDelimiter\abs{\lvert}{\rvert}

\newcommand{\expect}{\operatorname{\mathbb{E}}\expectarg}
\DeclarePairedDelimiterX{\expectarg}[1]{[}{]}{%
  \ifnum\currentgrouptype=16 \else\begingroup\fi
  \activatebar#1
  \ifnum\currentgrouptype=16 \else\endgroup\fi
}

\newcommand{\expectpi}{\operatorname{\mathbb{E}_{\pi}}\expectpiarg}
\DeclarePairedDelimiterX{\expectpiarg}[1]{[}{]}{%
  \ifnum\currentgrouptype=16 \else\begingroup\fi
  \activatebar#1
  \ifnum\currentgrouptype=16 \else\endgroup\fi
}

\newcommand{\innermid}{\nonscript\;\delimsize\vert\nonscript\;}
\newcommand{\activatebar}{%
  \begingroup\lccode`\~=`\|
  \lowercase{\endgroup\let~}\innermid 
  \mathcode`|=\string"8000
}


\makeatletter
\newcommand{\mybox}{%
    \collectbox{%
        \setlength{\fboxsep}{1pt}%
        \fbox{\BOXCONTENT}%
    }%
}


%%%%%%%%%%%%%%%% Direct Comments %%%%%%%%%%%%%%%%
\newcommand{\joe}[1]{\textcolor{yellow}{\colorbox{blue}{\parbox{15.5cm}{\textbf{\textsc{Joe}: \ #1}}}}}
%newcommand{\joe}[1]{}
%%%%%%%%%%%%%%%%%%%%%%%%%%%%%%%%%%%%%%%%%%

\begin{document}
\maketitle

\section*{\large Problem 1: Finding the value of a state under a policy}

\begin{proof}
\text{}\\\ \\\
First, using the law of total expectation and conditioning over $a \in \mathcal{A}$:
\begin{align*} 
v_{\pi}(s) &  = \ \expectpi{G_t  |  S_t = s } \ = \ \sum_{a} \expectpi*{  G_t   |   S_t = s, a_t = a }\cdot \underbrace{P\left(a_t = a | s_t = s\right)}_{ = \ \pi(a  |  s)} \\[0.5em]
& = \ \sum_{a} \pi(a \ | \ s)\expectpi*{  G_t   |   S_t = s, a_t = a } \\
\intertext{Again, using the law of total expectation, but now conditioning over $s'$:}
& = \ \sum_{a} \pi(a \ | \ s)\sum_{s'}\expectpi*{  G_t   |   S_t = s, a_t = a, S_{t+1} = s'} \cdot \underbrace{P\left(S_{t+1} = s' \ | \ S_t = s, a_t = a \right)}_{= \ P(s' | s, a)} \\[1em]
& = \ \sum_{a} \pi(a \ | \ s)\sum_{s'} P(s' \ | \ s, a)\expectpi*{  G_t   |   S_t = s, a_t = a, S_{t+1} = s'}\\[0.2em]
\intertext{Substituting in the definition of $G_t$:} 
& = \ \sum_{a} \pi(a \ | \ s)\sum_{s'} P(s' \ | \ s, a)\expectpi*{  \expect*{\sum_{k=1}^{\infty} \gamma^{k-1}R_{t+k}}   |   S_t = s, a_t = a, S_{t+1} = s'}\\[0.2em]
\intertext{Pulling out the first summand, $R_{t+1}$, from the infinite series:}
& = \ \sum_{a} \pi(a \ | \ s)\sum_{s'} P(s' \ | \ s, a)\expectpi*{  \expect*{R_{t+1} + \sum_{k=2}^{\infty} \gamma^{k-1}R_{t+k}}   |   S_t = s, a_t = a, S_{t+1} = s'}\\[0.2em]
\intertext{Re-indexing the infinite series and factoring out a $\gamma$ :}
& = \ \sum_{a} \pi(a \ | \ s)\sum_{s'} P(s' \ | \ s, a)\expectpi*{  \expect*{R_{t+1} + \gamma\sum_{k=1}^{\infty} \gamma^{k-1}R_{t+k+1}}   |   S_t = s, a_t = a, S_{t+1} = s'}\\[0.2em]
\intertext{Using the linearity property of expectation:}
& = \ \sum_{a} \pi(a \ | \ s)\sum_{s'} P(s' \ | \ s, a)\expectpi*{  R_{t+1} + \gamma\expect*{ \sum_{k=1}^{\infty} \gamma^{k-1}R_{t+k+1}}  |  S_t = s, a_t = a, S_{t+1} = s'}\\[1em]
& = \ \sum_{a} \pi(a \ | \ s)\sum_{s'} P(s' \ | \ s, a)\expectpi*{  R_{t+1} + \gamma G_{t+1}  |  S_t = s, a_t = a, S_{t+1} = s'}\\[1em]
\intertext{Again, using the linearity property of expectation:}
& = \ \sum_{a} \pi(a \ | \ s)\sum_{s'} P(s' \ | \ s, a)\bigg[\expectpi*{  R_{t+1} |  S_t = s, a_t = a, S_{t+1} = s'}  \\[-0.5em]
&\quad \ \ \ \ \ \ \ \ \ \ \ \ \ \ \ \ \ \ \ \ \ \ \ \ \ \ \ \ \ \ \ \ \ \ \ \ \ \ \ \ \ \ \ + \ \gamma\expectpi*{G_{t+1} | S_t = s, a_t = a, S_{t+1} = s'}\bigg] \\
\intertext{By the Markov property, $\expectpi*{G_{t+1} | S_t = s, a_t = a, S_{t+1} = s'} =  \expectpi*{G_{t+1} | S_{t+1} = s'}$. Applying this to the above:}
& = \ \sum_{a} \pi(a \ | \ s)\sum_{s'} P(s' \ | \ s, a)\bigg[\underbrace{\expectpi*{  R_{t+1} |  S_t = s, a_t = a, S_{t+1} = s'}}_{= \ R_a(s, s')}  \ + \ \gamma\underbrace{\expectpi*{G_{t+1} | S_{t+1} = s'}}_{= \ v_{\pi}(s')}\bigg] \\[0.5em]
& = \ \sum_{a} \pi(a \ | \ s)\sum_{s'} P(s' \ | \ s, a)\big[R_a(s, s') \ + \ \gamma v_{\pi}(s') \big] \\[0.5em]
\end{align*}
Hence, $$v_{\pi}(s) \ = \ \sum_{a} \pi(a \ | \ s)\sum_{s'} P(s' \ | \ s, a)\big[R_a(s, s') \ + \ \gamma v_{\pi}(s') \big]$$
\end{proof}
%\clearpage
%\paragraph{Alternative Proof for Problem 1.} 
%\text{}
%\\\ \\
%The following proof is more brief than \textit{my} proof above, and is inspired by the Bellman equation proof provided in \textit{Reinforcement Learning: An Introduction}, by Sutton.
%\begin{align*}
%v_{\pi}(s) &  = \ \expectpi{G_t  |  S_t = s } \ = \ \expectpi*{  \mathbb{E}\left[ \sum_{k = 1}^{\infty} \gamma^{k-1}R_{t+k}  \right]   |   S_t = s } \\[1em]
%& = \ \expectpi*{ \mathbb{E}\left[ R_{t+1} +  \gamma\sum_{k = 1}^{\infty} \gamma^{k-1}R_{t+k+1}  \right]  | S_t = s }\\[1em]
%& = \ \expectpi*{R_{t+1} +  \gamma G_{t+1} | S_t = s}\\[1em] 
%& = \ \expectpi*{R_{t+1} +  \gamma \expectpi{G_{t+1} | S_{t+1} = s'} | S_t = s}  && \text{(\emph{by law of total expectation})}\\[1em] 
%& = \  \expectpi*{R_{t+1} +  \gamma v_{\pi}(s') | S_t = s}  \\[1em]  
%& = \ \sum_{a} \pi(a \ | \ s) \sum_{s'} P\left(s' \ | \ s, a \right)\left[R_{a}(s, s') + \gamma v_{\pi}(s') \right] \\[1em]
%& = \ \expectpi*{\sum_{s'} P\left(s' | s, a \right)\left[R_{t+1}  +  \gamma\sum_{k = 1}^{\infty} \gamma^{k-1}R_{t+k+1}  \right] | S_t = s  }  && \text{(\emph{by def. of expected value})}\\[1em]
%& = \ 
%\end{align*}

\section*{\large Problem 2: Solving for a value function using linear algebra}

In Problem 1 it was shown that $$v_{\pi}(s) \ = \ \sum_{a} \pi(a \ | \ s)\sum_{s'} P(s' \ | \ s, a)\big[R_a(s, s') \ + \ \gamma v_{\pi}(s') \big]$$ 
Applying the assumption that all transitions are deterministic, i.e., $P(s' \ | \ s, a) = \mathbbm{1}\{s' = next(s,a)\}$, we get $$v_{\pi}(s) \ = \ \sum_{a} \pi(a \ | \ s)\sum_{s'} \mathbbm{1}\{s' = next(s,a)\}\cdot\big[R_a(s, s') \ + \ \gamma v_{\pi}(s') \big]$$ 
That is, for each action $ a $, given the current state $s$, there is exactly one subsequent state. Thus, for each action $a$, the second summation reduces to a single term:
\begin{align*}
v_{\pi}(s) \ & = \ \sum_{a} \pi(a \ | \ s)\big[R_a(s, s') \ + \ \gamma v_{\pi}(s') \big]\\[0.6em]
& = \ \sum_{a} \pi(a \ | \ s)R_a(s, s') \ + \ \gamma\sum_{a} \pi(a \ | \ s)v_{\pi}(s')\\[0.6em]
& = \ \expectpi*{R_a(s, s') | S_t = s} \ + \ \gamma\expectpi{v_{\pi}(s') | S_t = s}\\
\end{align*}
Rearranging, we have
\begin{align*}
v_{\pi}(s) \ - \ \gamma\expectpi{v_{\pi}(s') | S_t = s} \ & = \ \expectpi*{R_a(s, s') | S_t = s} \\[0.6em]
\Longrightarrow \ \ \ \ v_{\pi}(s) \ - \ \gamma \sum_{s'} P(s' | s)v_{\pi}(s')  \ & = \ \expectpi*{R_a(s, s') | S_t = s} \\[0.6em]
\Longrightarrow \ \ \ \ v_{\pi}(s) \ - \ \gamma \sum_{s'} P(s' | s)v_{\pi}(s')  \ & = \ \expectpi*{R_a(s, s') | S_t = s} \\[0.6em]
\intertext{where $P(s'|s)$ is the probability of transitioning from state $s$ to the subsequent state $s'$, and so $\sum_{s'}P(s' | s)$ is the sum of transition probabilities over all possible subsequent states.}
\intertext{The above can be expressed as a system of linear equations, where each equation is the value function at a particular current state $s_i$. All possible subsequent states are denoted by $s_j$ (see below).}
v_{\pi}(s_i) \ - \ \gamma \sum_{j=1}^n P(s_j | s_i)v_{\pi}(s_j)  \ & = \ \expectpi*{R_a(s, s') | S_t = s}
\end{align*}
In matrix form, this is
\begin{align*}
\begin{bmatrix} v_{\pi}(s_1) \\ \vdots \\ v_{\pi}(s_n) \end{bmatrix} \ -  \  \gamma\begin{bmatrix} p(s_1 | s_1) & \cdots & p(s_n | s_1) \\ \vdots & \ddots & \vdots \\ p(s_1 | s_n) & \cdots & p(s_n | s_n)  \end{bmatrix} \begin{bmatrix} v_{\pi}(s_1) \\ \vdots \\ v_{\pi}(s_n) \end{bmatrix} \ & = \ \begin{bmatrix} \expectpi*{R_a(s_1, s')} \\ \vdots \\ \expectpi*{R_a(s_n, s')} \end{bmatrix} 
\end{align*}
That is,  
\begin{align*}
\mathbf{v}_{\pi} \ - \ \gamma \mathbf{P} \mathbf{v}_{\pi} \ & = \ \mathbf{R}_{\pi}\\
\Longrightarrow \ \ \ \ \ \ \left(\mathbf{I} - \gamma \mathbf{P}\right)\mathbf{v}_{\pi} \ & = \ \mathbf{R}_{\pi}
\end{align*}
where $\mathbf{R}_{\pi}$ denotes the vector of expected returns on the right-hand side of the expression above.




%Now, applying this result for the current state $s_i$, for all $i$, and for each possible subsequent state $s_j$, $i, j \in \{1, \dots, n\}$, we get:
%$$v_{\pi}(s_i) \ = \ \sum_{j=1}^n \pi(a_i \ | \ s_i)R_a(s_i, s_j) \ + \ \gamma\sum_{j=1}^n \pi(a_i \ | \ s_i)v_{\pi}(s_j)$$
%
%where $a_{ij}$ denotes the action that results in transitioning to state $s_j$ from state $s_i$. 
%Putting this into matrix form:
%$$\begin{bmatrix} v_{\pi}(s_1) \\ \vdots \\ v_{\pi}(s_n) \end{bmatrix}  = \begin{bmatrix} \pi(a_{11} | s_1) & \cdots & \pi(a_{1n} | s_1) \\ \vdots & \ddots & \vdots \\ \pi(a_{n1} | s_n) & \cdots & \pi(a_{nn} | s_n)  \end{bmatrix} \begin{bmatrix} R_1 \\ \vdots \\ R_n \end{bmatrix} \ + \ \gamma  \begin{bmatrix} \pi(a_{11} | s_1) & \cdots & \pi(a_{1n} | s_1) \\ \vdots & \ddots & \vdots \\ \pi(a_{n1} | s_n) & \cdots & \pi(a_{nn} | s_n)  \end{bmatrix} \begin{bmatrix} v_{\pi}(s_1) \\ \vdots \\ v_{\pi}(s_n) \end{bmatrix}$$


\section*{\large Problem 3: Finding the value states ``in the real world"}

For uniform policy $\pi$ and $\gamma = 0.9$, it can be seen that the north is favored. In particular, it appears that state A is the optimal state to be in. Note the negative values along the bottom of the grid. This is because of the increased probability of leaving the grid under this policy. 

\begin{figure}[H]
\centering
\caption{Grid for Uniform Policy $\pi$ and $\gamma = 0.9$}
\includegraphics[scale=0.5]{"../code/figures/p3_part1"}
\label{fig:uniform-grid}
\end{figure}

For non-uniform policy $\pi(north|s) = 0.7$  and $0.1$, it can be seen that the north is favored here as well.  Again, it appears that state A is the optimal state to be in. Note again, the negative values along the bottom of the grid.

\begin{figure}[H]
\centering
\caption{Grid for $\pi(north|s) = 0.7$  and $0.1$ for all other directions}
\includegraphics[scale=0.5]{"../code/figures/p3_part2"}
\label{fig:uniform-grid}
\end{figure}

For non-uniform policy $\pi(north|s) = \pi(south|s) = 0.4$  and $\pi(east|s) = \pi(west|s) = 0.1$, one will note that, again, the north is favored.  Again, it appears that state A is the optimal state to be in. Note again, the negative values along the bottom of the grid. 
\begin{figure}[H]
\centering
\caption{Grid for $\pi(north|s) = \pi(south|s) = 0.4$  and $\pi(east|s) = \pi(west|s) = 0.1$}
\includegraphics[scale=0.5]{"../code/figures/p3_part3"}
\label{fig:uniform-grid}
\end{figure}





\section*{\large Problem 4: Finding an optimal value function}

\begin{enumerate}[(a)]
%\item  From problem 1, the value function can be written as $$v_{\pi}(s) \ = \ \sum_{a} \pi(a \ | \ s)\sum_{s'} P(s' \ | \ s, a)\big[R_a(s, s') \ + \ \gamma v_{\pi}(s') \big]$$
\item
Because a policy governs the actions taken and the associated probabilities, maximizing $v_{\pi}(s)$ for a given state $s$, over all policies $\pi$, is tantamount to choosing the action $a$ that achieves the maximum value.  Let $a^*$ denote such an action. Then, $a^* = \argmax_a\{v_{\pi}(s)\}$. Then, an optimal policy $\pi^*$, will always prescribe $a^*$. More specifically, for a given state $s$, an optimal policy will always distribute all of the weight/probability to $a^*$. Then, because $\sum_{a} \pi(a \ | \ s) = 1$, for a given $s$, we have that for an optimal policy $\pi^*$: $$\pi^*(a \ | \ s)  = \begin{cases} 1 & \text{if } a = \mathlarger{\argmax_a\{v_{\pi}(s)\}} \\ 0 & \text{otherwise} \end{cases}$$
In other words, $\pi^*$ is deterministic. The value function for an optimal policy, $\pi^*$, is such that 
\begin{align*}
v_{\pi^*}(s) & =  \ \sum_{a}\pi^*(a \ | \ s)\sum_{s'} P(s' \ | \ s, a)\big[R_a(s, s') \ + \ \gamma v_{*}(s') \big]\\
& =  \ \left(\pi^*(a_1 | s) + \cdots + \pi^*(a^* | s) + \cdots + \pi^*(a_k | s)\right)\sum_{s'} P(s' \ | \ s, a)\big[R_a(s, s') \ + \ \gamma v_{*}(s') \big] \\
& = \ \left(0 + \cdots + \ 1 \ + \cdots + 0 \right)\sum_{s'} P(s' \ | \ s, a)\big[R_a(s, s') \ + \ \gamma v_{*}(s') \big]\\
& = \ \sum_{s'} P(s' \ | \ s, a)\big[R_a(s, s') \ + \ \gamma v_{*}(s') \big]
\end{align*}
In $4b$ it's argued that $v_{\pi^*}(s) = v_*(s)$. The optimal value function is then
\begin{align*}
v_*(s) \ & = \ \max_{\pi}\{v_{\pi}(s)\} \\
%& = \ \max_{\pi} \ \expectpi*{G_t | S_t = s} \ = \ \sum_{\ \\
& = \ \max_{\pi} \left\lbrace \sum_{a} \pi(a \ | \ s)\sum_{s'} P(s' \ | \ s, a)\big[R_a(s, s') \ + \ \gamma v_{\pi}(s') \big]\right\rbrace\\[1em]
\intertext{By the recursive definition of the value function, the optimal value at a state $s$ must include the optimal value (discounted by $\gamma$) for each subsequent state, $s'$. It then follows that,}
& = \ \max_{a} \left\lbrace \sum_{a} \pi(a \ | \ s)\sum_{s'} P(s' \ | \ s, a)\big[R_a(s, s') \ + \ \gamma \max_{\pi} v_{\pi}(s') \big]\right\rbrace\\[1em]
\intertext{From the above argument that an optimal policy will always distribute all of the weight/probability to $a^*$:}
& = \ \max_{a} \left\lbrace \sum_{s'} P(s' \ | \ s, a)\big[R_a(s, s') \ + \ \gamma v_{*}(s') \big]\right\rbrace\\[1em]
\end{align*}
Therefore, $$v_*(s) \ = \ \max_{a} \sum_{s'} P(s' \ | \ s, a)\big[R_a(s, s') \ + \ \gamma v_{*}(s') \big]$$

\item 
%From part (a) we know that $v_*(s) = \mathlarger{\max_{a} \sum_{s'} P(s' \ | \ s, a)\big[R_a(s, s') \ + \ \gamma v_*(s') \big]}$. Based on this definition, 
An optimal policy can be found by recursively choosing the action that achieves the maximum value at each state $s$, and distributing all of the weight/probability to this action. That is, at each state $s$, choose the action $a^* = \argmax_{a} v_{\pi}(s)$, and set $\pi(a^*|s) = 1$ and assign $0$ to all other policies. Then at each state $s$, the value function for this optimal policy $\pi^*$ is such that \ $v_{\pi}(s) \leq v_{\pi^*}(s)$ for all policies $\pi$ and all $s \in \mathcal{S}$. However, this is also the relationship that $v_*(s)$ has with $v_{\pi}(s)$, for all policies $\pi$. Indeed, it follows from the definition of $v_*(s)$ being the optimal value function that $v_{\pi}(s) \leq v_{*}(s)$ for all policies $\pi$ and all $s \in \mathcal{S}$. Therefore, it must be the case that $v_{\pi^*}(s) = v_*(s)$. Thus, the optimal policy is indeed optimal.\\\

The optimal policy $\pi^*$ is deterministic. Indeed, as discussed in part (a), it will always choose the action $a^* = \mathlarger{\argmax_{a} v_{\pi}(s)}$. Therefore, all of the probability/weight will be allocated to $a^*$ at each state, whereas all other actions will be assigned a probability of zero. This implies that the optimal policy is non-random. That is, the action it prescribes when in state $s$ is a deterministic function of $s$.

\item \begin{proof}
First, note that \ $q_{\pi}(s, a)  \ = \ \expectpi*{G_t | S_t = s , A_t = a}$. Now, starting from the definition of the value function, we have
\begin{align*}
v_{\pi}(s) \ & = \ \expectpi*{G_t | S_t = s} \\[0.5em]
& = \ \sum_{a} \underbrace{\expectpi*{  G_t   |   S_t = s, A_t = a }}_{= \ q_{\pi}(s,a)}\pi(a \ | \ s) & \textit{by the law of total expectation}\\[0.5em]
& = \ \sum_{a} \pi(a \ | \ s)q_{\pi}(s,a)
\end{align*}
That is, $$v_{\pi}(s) \ = \ \sum_{a} \pi(a \ | \ s)q_{\pi}(s,a)$$

In Problem 1, we found that $$v_{\pi}(s) \ = \ \sum_{a} \pi(a \ | \ s)\sum_{s'} P(s' \ | \ s, a)\big[R_a(s, s') \ + \ \gamma v_{\pi}(s') \big]$$
Therefore, 
\begin{align*}
\sum_{a} \pi(a \ | \ s)q_{\pi}(s,a) \ & = \ \sum_{a} \pi(a \ | \ s)\sum_{s'} P(s' \ | \ s, a)\big[R_a(s, s') \ + \ \gamma v_{\pi}(s') \big]\\[1em]
\Longrightarrow \ \ \ \ q_{\pi}(s,a) \ & = \ \sum_{s'} P(s' \ | \ s, a)\big[R_a(s, s') \ + \ \gamma v_{\pi}(s') \big]
\end{align*}
Using this form of the $q$-function, we will maximize over all policies $\pi$:
\begin{align*}
q_*(s,a) \ = \ \max_{\pi} q_{\pi}(s,a) \ & = \ \max_{\pi} \sum_{s'} P(s' \ | \ s, a)\big[R_a(s, s') \ + \ \gamma v_{\pi}(s') \big]\\[0.5em]
& = \ \max_{\pi} \left\lbrace \sum_{s'} P(s' \ | \ s, a)R_a(s, s') \ + \ \gamma \sum_{s'} P(s' \ | \ s, a) v_{\pi}(s') \right\rbrace \\[0.5em]
\intertext{Since we are maximizing over $\pi$, all terms independent of $\pi$ can be moved outside of the max operator. While actions are dependent on $\pi$, the terms being moved outside of the max operator are conditioning on the action $a$. That is, $a$ is given in those terms:}
& = \ \sum_{s'} P(s' \ | \ s, a)R_a(s, s') \ + \ \gamma \sum_{s'} P(s' \ | \ s, a) \max_{\pi} v_{\pi}(s') \\[0.5em]
& = \ \sum_{s'} P(s' \ | \ s, a)R_a(s, s') \ + \ \gamma \sum_{s'} P(s' \ | \ s, a) v_{*}(s') \\[0.5em]
& = \ \sum_{s'} P(s' \ | \ s, a)\big[R_a(s, s') \ + \ \gamma v_{*}(s')\big] \\[0.5em]
\Longrightarrow \ \ \ \ q_*(s, a) \ & = \ \sum_{s'} P(s' \ | \ s, a)\big[R_a(s, s') \ + \ \gamma v_{*}(s')\big] \\[0.5em]
\intertext{Now, substituting in the expression for $v_*$ from $4a$ into $q_*(s,a)$:}
 q_*(s, a) \ & = \ \sum_{s'} P(s' \ | \ s, a)\big[R_a(s, s') + \gamma \max_{a'} {\color{blue} \underbrace{\color{black} \sum_{s''} P(s'' \ | \ s', a')[R_{a'}(s', s'') + \gamma v_{*}(s'')]}_{= \ q_*(s', a')}}\big]
\intertext{where $s''$ denotes the state subsequent to $s'$. Notice that the expression in the max operator satisfies the equation for $q_*$ that was determined above. Hence, we have}
 q_*(s, a) \ & = \ \sum_{s'} P(s' \ | \ s, a)\big[R_a(s, s') + \gamma \max_{a'}  q_*(s', a')\big]
\end{align*}

\end{proof}

\end{enumerate}

\section*{\large Problem 5: Finding the optimal policy using iterative methods}
\begin{proof}
\begin{align*}
\infnorm{V_t(s) - V_*(s)} \ & = \ \max_{s \in \mathcal{S}} \ \lvert V_t(s) - V_*(s) \rvert \\[0.5em]
& = \ \max_{s \in \mathcal{S}} \ \abs[\bigg]{\max_{a} \sum_{s'} P(s' | s, a)\big[R_a(s, s') \ + \  \gamma V_{t-1}(s')\big] \\[-0.9em]
&\quad \ \ \ \ \ \ \ \ \ \ \ \ \ \ \ \ \ \ \ \ \ \ \ \ \ \ \ \ \ \ \ \ \ \ \ \ \ \ \ \ \ \ \ \ \ \ \ - \ \max_{a} \sum_{s'} P(s' | s, a)[R_a(s, s') \ + \ \gamma V_*(s')]} \\[0.5em]
& \leq \ \max_{s \in \mathcal{S}} \ \abs[\Bigg]{\max_{a} \left\lbrace \sum_{s'} P(s' | s, a)\big[R_a(s, s') \ + \ \gamma V_{t-1}(s')\big] \right. \\[-0.9em]
&\quad \ \ \ \ \ \ \ \ \ \ \ \ \ \ \ \ \ \ \ \ \ \ \ \ \ \ \ \ \ \ \ \ \ \ \ \ \ \ \ \ \ \ \ \ \ \ \ \left. - \ \sum_{s'} P(s' | s, a)\big[R_a(s, s') \ + \ \gamma V_*(s')\big]\right\rbrace} \\[0.5em]
& = \ \max_{s \in \mathcal{S}} \ \abs[\bigg]{\max_{a} \underbrace{\sum_{s'} P(s' | s, a)}_{= \ 1}\big[\gamma V_{t-1}(s')  \ - \ \gamma V_*(s')\big]} \\[1em]
& = \ \gamma\max_{s \in \mathcal{S}} \ \abs[\big]{V_{t-1}(s') \ - \ V_*(s')} \\[0.5em]
& \leq \ \gamma\max_{s \in \mathcal{S}} \ \abs[\big]{V_{t-1}(s)  \ - \ V_*(s)} \\[0.5em]
& = \ \gamma\infnorm{V_{t-1}(s) - V_*(s)}
\end{align*}
Therefore,  $$\infnorm{V_t(s) - V_*(s)} \ \leq \ \gamma\infnorm{V_{t-1}(s) - V_*(s)}$$
In general, using the same argument as above, it can be shown that for all $k \in \{1, 2, \dots, t-1\}$ and any $\gamma \in (0, 1)$
$$\infnorm{V_{t-k}(s) - V_*(s)} \ \leq \ \gamma\infnorm{V_{t-k-1}(s) - V_*(s)}$$
Then,
\begin{align*}
\infnorm{V_t(s) - V_*(s)} \ & \leq \ \gamma\infnorm{V_{t-1}(s) - V_*(s)}\\
& \leq \ \gamma^2\infnorm{V_{t-2}(s) - V_*(s)}\\
& \leq \ \gamma^3\infnorm{V_{t-3}(s) - V_*(s)}\\
& \ \ \vdots \ \ \ \ \ \ \ \ \ \ \ \ \ \ \ \vdots \\
& \leq \ \gamma^{t-1}\infnorm{V_{1}(s) - V_*(s)}\\
& \leq \ \gamma^t\infnorm{V_0(s) - V_*(s)}\\
\end{align*}
$\Longrightarrow \ \ \ \ \infnorm{V_t(s) - V_*(s)} \ \leq \ \gamma^t\infnorm{V_0(s) - V_*(s)}$
\end{proof}
\noindent
\textit{Now to show that the inequality above implies convergence.}\\\ \\\
\noindent
Since $\gamma \in (0, 1), \ \ \gamma^t \longrightarrow 0 \ $ as $ \ t \longrightarrow \infty$\\\ \\\
$\Longrightarrow \ \ \ \gamma^t\infnorm{V_0(s) - V_*(s)} \longrightarrow 0 \ $ as $ \ t \longrightarrow \infty$\\\ \\\
Then, the inequality \ $\infnorm{V_t(s) - V_*(s)} \ \leq \ \gamma^t\infnorm{V_0(s) - V_*(s)}$ \  implies that \\\ \\\
$\infnorm{V_t(s) - V_*(s)} \longrightarrow 0 \ $ as $ \ t \longrightarrow \infty$\\\ \\\
Therefore, the value function converges to the optimal value function as $t \longrightarrow \infty$.

\section*{\large Problem 6: Find the optimal value function for gridworld}

The optimal value function was computed using the value iteration approach (see below). This optimal policy in this case is conceptually sound given that the greatest reward is in A.
\begin{figure}[H]
\centering
\caption{Optimal Value Grid}
\includegraphics[scale=0.5]{"../code/figures/p3_part4"}
\label{fig:uniform-grid}
\end{figure}

\section*{\large Problem 7: A model-free approach}
\begin{enumerate}[i)]
\item At every time step, the $Q$ estimate changes. In particular, it increases or decreases depending on the magnitude of the reward associated with the action taken. For larger rewards, there is a larger change in the $Q$ estimate, while for smaller rewards, there is a smaller change in the  estimate of $Q$. However, by implementing an $\epsilon$-greedy policy and for a reasonable value of $\alpha$, the changes in the $Q$ estimate between states will diminish. Indeed, if $\alpha$ is sufficiently small, your policy will learn on the most up-to-date information available. Therefore, for such an $\alpha$ and by acting $\epsilon$-greedily with respect to the $Q$ estimate, the policy will improve at every timestep. As the policy improves, your estimate of $Q$ improves, and since the change in the estimates decrease at every timestep, the $Q$-estimate will converge to the optimal state action function for large values of $t$ (Recall the result from Problem 5).

\item Despite having started the homework early, I was not able to complete $7$(ii) in time. However, I am actively working on this problem, so that I don't miss out on anything. Sorry to disappoint. 
\end{enumerate}

\section*{Extra Credit}
\begin{enumerate}[i)] 
\item  From the definition of $G_t$, we have
\begin{align*} 
G_t \ & = \ \expect*{\sum_{n=1}^{\infty} \gamma^{n-1}R_{t+n}} \ = \ \expect*{R_{t+1} \ + \ \gamma\sum_{n=1}^{\infty} \gamma^{n-1}R_{t+n+1}} \\[1em]
& = \ \expect*{R_{t+1}} \ + \ \gamma\expect*{\sum_{n=1}^{\infty} \gamma^{n-1}R_{t+n+1}}\\[1em]
& = \ R_{t+1}  + \gamma G_{t+1}
\end{align*}
We also know that $V_{\pi}(S_t) = \expect*{G_t | S_t = s}$. Because, $V_{\pi}(S_t)$ is the expected value of $G_t$, conditional on a specific state,  $G_t$ can be estimated by $V_{\pi}(S_t)$ (biased estimate). Then,
$$G_t \ \approx \ R_{t+1}  + \gamma V_{\pi}(S_{t+1})$$
The recursive rule then becomes 
$$V_{\pi}(S_t) \longleftarrow V_{\pi}(S_t) + \alpha[R_{t+1}  + \gamma V_{\pi}(S_{t+1}) - V_{\pi}(S_t)]$$
At two timesteps ahead, $G_t$ can be expressed using following recursive argument:
\begin{align*}
G_t & = R_{t+1}  + \gamma G_{t+1}\\
& = R_{t+1}  + \gamma(R_{t+2}  + \gamma G_{t+2})\\
& = R_{t+1}  + \gamma R_{t+2}  + \gamma^2 G_{t+2}
\end{align*}
Now, since $G_{t+2}$ can be estimated by $V_{\pi}(S_{t+2})$, the $2$-steps-ahead estimate of $G_t$ is
$$G_t \approx R_{t+1}  + \gamma R_{t+2}  + \gamma^2 V_{\pi}(S_{t+2})$$
and so, the recursive rule for two timesteps ahead becomes
$$V_{\pi}(S_t) \longleftarrow V_{\pi}(S_t) + \alpha[R_{t+1}  + \gamma R_{t+2}  + \gamma^2 V_{\pi}(S_{t+2})- V_{\pi}(S_t)]$$

\item We can write $G_t$ in terms of the expected reward $n$-steps ahead using the same argument for two steps ahead in part (i):
\begin{align*}
G_t & = R_{t+1}  + \gamma G_{t+1}\\
& = R_{t+1}  + \gamma(R_{t+2}  + \gamma G_{t+2})\\
& = R_{t+1}  + \gamma R_{t+2}  + \gamma^2 G_{t+2}\\
& = R_{t+1}  + \gamma R_{t+2}  + \gamma^2 (R_{t+3}  + \gamma G_{t+3})\\
& = R_{t+1}  + \gamma R_{t+2}  + \gamma^2 R_{t+3}  + \gamma^3 G_{t+3}\\
& \ \  \vdots \\
& = R_{t+1}  + \gamma R_{t+2}  + \gamma^2 R_{t+3}  + \cdots + \gamma^{n-1} R_{t+n} + \gamma^n G_{t+n}\\
& = \sum_{k=1}^{n}\gamma^{k-1} R_{t+k} + \gamma^n G_{t+n} 
\end{align*}
By estimating $G_{t+n}$ with $V_{\pi}(S_{t+n})$, the $n$-steps-ahead estimate of $G_t$ is 
$$G_t \ \approx \ \sum_{k=1}^{n}\gamma^{k-1} R_{t+k} + \gamma^n V_{\pi}(S_{t+n})$$
The recursive update rule for the $n$-steps-ahead estimate is then
$$V_{\pi}(S_t) \longleftarrow V_{\pi}(S_t) + \alpha\left[\sum_{k=1}^{n}\gamma^{k-1} R_{t+k}  + \gamma^n V_{\pi}(S_{t+n}) - V_{\pi}(S_t)\right]$$

\item Our $n$-steps-ahead estimate of $G_t$ is 
$$G_t \ \approx \ \sum_{k=1}^{n}\gamma^{k-1} R_{t+k} + \gamma^n V_{\pi}(S_{t+n})$$

The sum of discounted rewards, $R_{t+k}$,  adds noise/variance to the estimate of $G_t$ since each term is non-deterministic (or random), and because there is randomness in each transition from state-to-state since the action chosen is also random. 
Then, because $G_t$ incurs noise from each additional transition (i.e., each additional $\gamma^kR_{t+k}$ term, $k \in \{2, ..., n\}$), the variance will increase as $n$ increases.

$V_{\pi}(S_{t+n})$ introduces bias into the estimate of $G_t$ since it is an estimate of the value function, $v_{\pi}$, at time step $t+n$. However, there is a discount factor $\gamma^n$ on $V_{\pi}(S_{t+n})$ in the estimated expression for $G_t$, where $\gamma \in (0, 1)$. Then, as $n$ increases, $\gamma^nV_{\pi}(S_{t+n})$ decreases, and so the bias in the estimate decreases.

\item Set $a_n = (1 - \lambda)\lambda^n$ for $0 < \lambda < 1$. Indeed, $\lambda^n$ gives less weight to the $G_t^{(n)}$ for larger values of $n$, making the $G_t^{(n)}$ worth exponentially less for larger values of $n$. The $(1 - \lambda)$ ensures that the weighted sum is a convex combination. Indeed, 
\begin{align*}
(1 - \lambda)\sum_{n=1}^{n}\lambda^n & = (1 - \lambda)(1 + \lambda + \lambda^2 + \cdots + \lambda^n)\\
& = 1 - \lambda^n \longrightarrow 1 \ \text{as} \ n \longrightarrow \infty \\
\intertext{That is,} 
(1 - \lambda)\sum_{n=1}^{\infty}\lambda^n & = 1
\end{align*}
The recursive update rule is then,
$$V_{\pi}(S_t) \longleftarrow V_{\pi}(S_t) + \alpha\left[(1 - \lambda)\sum_{n=1}^{\infty}\lambda^n G_t^{(n)}  - V_{\pi}(S_t)\right]$$

\item This approach requires that all $n$ returns be determined before updating the value function estimate. Therefore, the algorithm makes updates every $n$-steps, or at the end of each episode.


\item Because $\lambda \in (0, 1)$,  the $(1-\lambda)\lambda^n$ weights on $G_t^{(n)}$ are exponentially smaller for smaller values of $\lambda$, decreasing the impact the noise has on each $k$-steps-ahead estimate coming from the rewards and transitions/actions. Similarly, the weights are exponentially larger for larger values of $\lambda$, increasing the impact the noise has on each estimate. Therefore, the larger the value of $\lambda$, the larger the variance. On the other hand, larger values of $\lambda$ will result in smaller bias. 

\end{enumerate}

\end{document} 
